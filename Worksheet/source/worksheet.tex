\documentclass[11pt,a4paper]{scrartcl}
\typearea{12}
\usepackage{graphicx}
\usepackage{pstricks}
\usepackage{listings}
\lstset{language=python}
\pagestyle{headings}
\markright{Integrate and fire - worksheet}
\begin{document}

\subsection*{Worksheet}

This is a computational worksheet on the integrate and fire neuron; it
is not for marking or exam but if you work on it it will help you
understand integrate and fire neurons. I am happy to look at attempts
or to talk you through the challenges, just send me an email
\texttt{conor.houghton@bristol.ac.uk}.


\subsubsection*{Solving differential equations with Euler's method}

\begin{enumerate}

\item Solve numerically in Python using Euler's method the differential equation
\begin{equation}
\frac{df}{dt}=f^2-3f+e^{-t}
\end{equation}
on the interval $[0,3]$ with time step $\delta t=0.01$ and graph the
solution, taking care to label the axes. Although Python has good
libraries for solving differential equations numerically it would be
useful educationally not to use them for these question.

\item For the problem above try solve with $\delta t=0.01$, $0.1$,
  $0.5$ and one. Plot all the curves on one graph. What is a good
  value of $\delta t$ for this equation.

\end{enumerate}

\subsubsection*{Integrate and fire neurons}

\begin{enumerate}

\item Simulate an integrate and fire model with the following
  parameters for 1 s: $\tau_m = 10 $ms, $E_L = V_r = -70$ mV, $V_t =
  -40$ mV, $R_m= 10$ M$\Omega$, $I_e = 3.1 $ nA. Use Euler's method
  with timestep $\delta t = 1$ ms. Here $E_L$ is the leak potential,
  $V_r$ is the reset voltage, $V_t$ is the threshold, $R_m$ is the
  resistance and $\tau_m$ is the membrane time constant. Plot the
  voltage as a function of time. For simplicity assume that the neuron
  does not have a refractory period after producing a spike. [20\% of
    marks]. You do not need to plot spikes - once membrane potential
  exceeds threshold, simply set the membrane potential to $V_r$.

\item Compute analytically the minimum current $I_e$ required for the
  neuron with the above parameters to produce an action
  potential.

\item Simulate the neuron for 1 s for the input current with amplitude
  $I_e$ which is 0.1 [nA] lower than the minimum current computed
  above, and plot the voltage as a functions of time.

\item Simulate the neuron for 1s for currents ranging from 2 [nA] to 5
  [nA] in steps of 0.1 [nA]. For each amplitude of current count the
  number of spikes produced, that is the firing rate. Plot the firing
  rate as the function of the input current. It is possible to
  calculate this curve analytically; you might find it interesting to
  try.

\item Simulate two neurons which have synaptic connections between
  each other, that is the first neuron projects to the second, and the
  second neuron projects to the first. Both model neurons should have
  the same parameters: $\tau_m = 20$ ms, $E_L = -70$ mV $V_r = -80$ mV
  $V_t = -54$ mV $R_mI_e = 18$ mV and their synapses should also have
  the same parameters: $R_m \bar{g}_s = 0.15$, $P = 0.5$, $\tau_s= 10$
  ms. For simplicity take the synaptic conductance to satisfy
\begin{equation}
\tau_s\frac{dg_s}{dt}=-g_s
\end{equation}
with a spike arriving causing $g_s$ to increase by $\bar{g}_sP$. Simulate two cases: a) assuming that the synapses are excitatory with $E_s = 0$ mV, and b) assuming that the synapses are inhibitory with $E_s = -80$ mV. For each simulation set the initial membrane potentials of the neurons $V$ to different values chosen randomly from between $V_r$ and $V_t$ and simulate one second of activity.

\item In many real neurons the firing rate falls off after the first few spikes. This can be simulated with a slow potassium current. For the neuron described in the first question add a slow potassium current. This current should have reversal potential $E_K=-80$ mV, its conductance should increase by 0.005 $($M$\Omega)^{-1}$ every time there is a spike, otherwise it should decay towards zero with time constant $\tau=200$ ms. Plot the voltage of this neuron for one second.

\end{enumerate}

\end{document}

