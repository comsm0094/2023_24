\documentclass{article}
\usepackage[utf8]{inputenc}
\usepackage{graphicx}
\usepackage{fancyhdr}
\pagestyle{fancy}
\lfoot{\texttt{comsm0094.github.io}}
\lhead{LC\&B - 03.2 I\&F - Conor}
\rhead{\thepage}
\cfoot{}
\begin{document}
\subsection*{The leaky integrate and fire model}
This note is about the dynamics of a single neuron, it will cover
one of the simplest such models: the leaky integrate and fire model.


\begin{figure}[bh]
\begin{center}
  \includegraphics[width=6cm]{ion_pump_1.png}\\
  \includegraphics[width=6cm]{ion_pump_2.png}
\end{center}
\caption{The ion pump pumps three sodium ions out for each it two potassiums it pumps in.\label{fig:ion}}
\end{figure}


\subsection*{Electrical properties of a neuron}
The potential inside a neuron is lower than the potential on the
outside; this difference is created by ion pumps, small molecular
machines that use energy to pump ions across the membrane seperating
the inside and outside of the cell. One typical ion pump is
Na+/K+-ATPase (Sodium-potassium adenosine triphosphatase); this uses
energy in the form of ATP, the energy carrying molecule in the body,
and through each cycle it moves three sodium ions out of the cell and
two potassium ions into the cell, see Fig.~\ref{fig:ion} for a cartoon. Since both sodium and potassium ions
have a charge of plus one, this leads to a net loss of one atomic
charge to the inside of the cell lowering its potential. It also
creates an excess of sodium outside the cell and an excess of
potassium inside it. We will return to these chemical imbalances
later. The potential difference across the membrane is called the
\textbf{membrane potential}. At rest a typical value of the membrane
potential is $E_L=-70 $mV. It is useful to remember that the excessive
sodium is outside the cell and potassium inside; I think of islands
which are surrounded by salty water, as in Fig.~\ref{in_out}.


\begin{figure}
\begin{center}
\includegraphics[width=6cm]{uk.png}
\end{center}
\caption{A cartoon to help you remember that there is more sodium outside the cell and more potassium inside.\label{in_out}}
\end{figure}


\subsection*{Spikes}

So the summary version of what happens in neuons is that
\textbf{synapses} cause a small increase or decrease in the voltage;
\textbf{excitatory synapses} cause an increase, \textbf{inhibitory
  synapses} a decrease. This drives the internal voltage dynamics of
the cell, these dynamics are what we will learn about here. If the
voltage exceeds a threshold, say $V_T=-55$ mV there is a nonlinear
cascade which produces a \textbf{spike} or \textbf{action potential},
a spike in voltage 1-2 ms wide which rises above 0 mV before, in the
usual description, falling to a reset value of $V_R=-65$ mV, the cell
then remains unable to produce another spike for a \textbf{refractory
  period} which may last about 5 ms. We will examine how spikes are
formed later, this involves the nonlinear dynamics of ion channels in
the membrane; first though we will consider the integrate and fire
model which ignores the details of how spikes are produced and
simplifies the voltage dynamics.

\subsection*{The bucket-like equation for neurons}

We will now try to extend the bucket-like equation we looked at before
so that it applies to neurons. First off we replace $h$, the height of
the water, by $V$ the voltage in the cell and $C$ will be replaced by
$C_m$, the capacitance of the membrane, the amount of electrical
charge that can be stored at the membrane is $C_mV$. The amount of
electrical charge is the analogue of the volume of water. Thus,
voltage is like height, charge is like the amount of water.


\begin{figure}[tbph]
\begin{center}
  \includegraphics[width=5cm]{children1.png}
  \includegraphics[width=5cm]{children2.png}\\
  \includegraphics[width=5cm]{children3.png}
  \includegraphics[width=5cm]{children4.png}
\end{center}
\caption{In the first panel the children are in the circle, however
  random movement quickly disperses them across the playground, in the second panel;
  however, in the final two panels the force exerted on the children by the musical dog
  counteracts the diffusion pressure; in the neuron, the ions are
  children and the role of snoppy is played by
  voltage.\label{fig:children}}
\end{figure}


The leak is a bit more complicated, because of the chemical gradients,
that is the effects of the differing levels of ions inside and outside
the cell along and their propensity to diffuse, the voltage at which
there is no leaking of charge is not zero, it is $E_L=-70 $mV,
roughly. This is an important aspect of how neurons behave, and one we
will encounter again looking at the Hodgkin-Huxley equation: you might
at first expect that if the voltage inside the cell was, say, -60 mV
then even if there was a high conductivity for potassium at the
membrane, the potassium ions would stay in the cell: they are positive
ions after all and so a negative voltage means the electrical force is
attracting them to the inside of the cell. However, this isn't quite
what happens, there is a high concentration of potassium inside the
cell and because of the random motion of particles associated with
temperature, these have a tendency to diffuse, that is to increase the
entropy of the situation by spreading out. It takes a force to
counteract this; see Fig.~ref{fig:children} for a whimsical illustration of this. This is the reversal potential, $E_L$, the voltage
required for zero current even if there is some conductivity. It turns
out that the normal Ohm's law applies around the reversal potential so
that the current out of the cell is proportional to $V-E_L$.

$G$ is now $G_m$, a conductance, measuring the porousness of the
membrane to the flow of ions, in other words, it gives the constant of proportionality for the leak current: the leak current out of the cell is
$G_m(V-E_L)$. We actually divide across by the conductance, and write
$R_m=1/G_m$, the resistance. Finally, we write $\tau_m=C_m/G_m$ to get
\begin{equation}
\tau_m\frac{dV}{dt}=E_L-V+R_mI
\end{equation}
$I$ might end up being synaptic input, but traditionally we write the
equation to match the \textsl{in vivo} experiment where $I$ is an
injected current from an electrode, so we write $I_e$, \lq{}e\rq{} for
electrode. $\tau_m$ is a time constant, using the notation of
dimensional analysis we have $[\tau_m]=T$. To check this note that the
units of capacitance are charge per voltage: $[C_m]=QV^{-1}$, the
units of resistance is voltage per current $[R_m]=VI^{-1}$ and current
is charge per time, $[I]=QT^{-1}$ so $[C_mR_m]=T$, time.

The equation above leaves out the possibility that there are other
non-linear changes in the currents through the membrane as $V$
changes. This is a problem since there are other non-linear changes in
the currents through the membrane as $V$ changes. The equation above
leaves these out, in fact, the nonlinear effects are strongest for
values of $V$ near where a spike is produced, so one approach is to
use the linear equation unless $V$ reaches a threshold value and then
add a spike \lq{}by hand\rq{}. This has the effect of changing the
voltage to a reset value, this mimics what happens in the neuron, or
in the Hodgkin Huxley model which we will look at next and which
includes the full non-linear dynamics which makes the spike. Anyway,
in summary
\begin{itemize}
\item $V$ satisfies
\begin{equation}
\tau_m\frac{dV}{dt}=E_L-V+R_mI_e
\end{equation}
\item If $V\ge V_T$ a spike is recorded and the voltage is set to a
  reset value $V_R$.
\end{itemize}
The reset value, the voltage after the spike is often set equal to the
 leak potential. This is the \textbf{leaky integrate and fire model}, a surprisingly
old model first introduced in \cite{Lapicque1907a}. It lacks lots of
the details important in the dynamics of neurons, but is useful and
often used for modeling the behavior of large neuronal networks or for
exploring ideas about neuronal computation in a relatively
straight-forward setting.

This model is easy to solve; if $I_e$ is constant we have already
solved it above up to messing around with constants:
\begin{equation}
V(t)=E_L+R_mI_e+[V(0)-E_L-R_mI_e]e^{-t/\tau_m}
\end{equation}
If $I_e$ is not
constant it may still be possible to solve the equation, but in any
case the equation can be solved numerically on a computer. An example
in given in Fig.~\ref{v_i_f}.

\begin{figure}
\begin{center}
\include{v_i_f}
\end{center}
\caption{An integrate and fire neuron with different inputs. For
  $R_mI=12 $mV the voltage relaxes towards the equilibrium value
  $V=E_L+R_mI_e=-58$ mV. It never reaches the threshold value of
  $V_T=-55 $mV. For $R_mI=16$ mV the voltage reaches threshold and so
  there is a spike; the spike is added by hand, in this case by
  setting $V$ to $20$ mV for one time step. The voltage is then
  reset. Here $\tau_m=10$ ms.\label{v_i_f}}
\end{figure}

One thing to notice is that there are no spikes for low values of the current. Looking at the equation 
\begin{equation}
\tau_m\frac{dV}{dt}=E_L-V+R_mI_e
\end{equation}
so the equilibrium value for constant $I_e$, the value where $V$ stops changing, is
\begin{equation}
\bar{V}=E_L+R_mI_e
\end{equation}
Now if this value $\bar{V}>V_T$ then as the neuron voltage increased
towards its equilibrium value, $\bar{V}$, it would reach the
threshold, $V_T$, and spike. Hence, if $\bar{V}>V_T$ the neuron will
spike repeatedly.  However if $\bar{V}<V_T$ then the neuron will not
spike for that input because it will never reach threshold. We won't do it here\footnote{This calculation is described in \texttt{note\_on\_the\_f-I\_curve.pdf}}, but, in fact, since we can
solve the equations for constant $I_e$ we can work out the $f-I$
curve, the relationship between the firing rate and the input
current. It is plotted in Fig.~\ref{f_i_curve}.


\begin{figure}
\begin{center}
\include{f_i_curve}
\end{center}
\caption{The firing rate, that is spikes per second, for the integrate
  and fire neuron with different constant inputs with $\tau_m=10$ ms,
  $V_T=-55$ mV and both the leak and reset given by $-70$ mV. Notice
  how there is no firing until a threshold is reached and after that
  the firing increases very quickly. \label{f_i_curve}}
\end{figure}

\subsection*{Beyond the integrate and fire equations}

The leaky integrate and fire model is very useful; from a practical
point-of-view it can be used in modelling, either for modelling large
networks of spiking neurons for neuroscience, for example, when
examinging the statistics of spiking or synapse strengths under
various regimes, or when investigating spiking neural networks in
machine learning. It also clarifies the role of the membrane time
constant, measuring this is a useful and important step in the
experimental exploration of a neuron's properties. From a theoretical
point-of-view; it gives important insight into what a neuron does.

The model has limitations; it does not incorporate the dynamics of the
active ion channels. Ultimately, this means it does not model the
production of the spike; in the integrate and fire model the spike is
included using the threshold condition, when the voltage exceeds the
threshold a spike is said to occur, but there is no indication of how
that happens. The \textbf{Hodgkin-Huxley equation} includes some
active channels and will demonstate the dynamics that can produce a
spike. However, it is possible to consider other, smaller,
modifications of the leaky integrate and fire model.

One issue with the leaky integrate and fire model is that it does not
model behaviour near the threshold, even before the neuron starts to
spike, when its voltage approaches the threshold where a spike becomes
inevitable, some of the active channels do start to open altering the
dynamics. Since this only applies very close to threshold it might not
seem a significant drawback of the model; however, in some
descriptions of neuronal networks some homeostatic principles of
balanced excitation and inhibition mean that neurons are often near
threshold making the dynamics near threshold important. There are
extensions to the leaking integrate and fire model that seek to
address this, usually by adding extra non-linear terms to the
differential equation; one successful example is the adaptive
exponential leaking integrate and fire model which adds an exponential
term.

Another more straight-forward difference between the
integrate-and-fire neuron and real neurons is refractoriness; most
neurons are unable to fire directly after a spike, there is a rich
discription of this \textbf{refactory period}, many neurons have a
short period where they are unable to spike and another, longer,
period where spike is less likely to occur. The refractory period is
often taken to be around 10 ms, but this does vary a lot between
neurons, Purkinje cells for example seem to have little or no
refractory period. This refactoriness is related to the active
channels, for example, during spiking potassium channels open, as we
will see this is part of the dynamics which produces the down-swing of
the spike and, while they are open, these potassium channels will push
the voltage towards the potassium reversal potential. This affect can
easily be added to the integrate-and-fire model, for example, by
adding to the spiking rule to stipulate that the voltage remains fixed
that the reset value for some specified time after a spike.

One final example is \textbf{spike rate adaptation}, in the
integrate-and-fire model a neuron with a constant input will spike at
a constant rate: in fact, the model resets completely with every spike
so the neuron never shows any effect of its previous behaviour. Real
neurons often show more complex behaviour, often, for example, a
neuron will spike slower as it continues to spike, or its second spike
will happen very easily, something called pair-pulse excitation but
will then slow down. There are even neurons, such as the mossy cells
of the hilar region of dentate gyrus, which spike faster and faster
the longer they are excited. This can be easily incorporated into the
integrate-and-fire neuron using an extra current:
\begin{equation}
\tau_m\frac{dV}{dt}=E_L-V+R_mI_e-W
\end{equation}  
where $W$ is itself a current which decays to zero:
\begin{equation}
\tau_w\frac{dW}{dt}=-W
\end{equation}  
and has its own rule for what happens when there is a spike:
\begin{equation}
  W\rightarrow \delta W
\end{equation}
Thus, if $\delta W$ is positive, $W$ will increase if there is a
spike, this gives a negative contribution to $dV/dt$, slowing down
spiking. $W$ then decays to zero with a timescale $\tau_w$. These
dynamics would model spike rate adaptation and the $W$ current itself
can be thought of as the effect of a slow potassium current. In fact,
spike rate adaptation is not explained by the Hodgkin-Huxley, but a
slow potassium current could be added there as well.









\begin{thebibliography}{99}
\bibitem{Lapicque1907a}
Lapicque, L. (1907). 
\newblock Recherches quantitatives sur l'excitation \'{e}lectrique des nerfs trait\'{e}e comme une polarisation. 
\newblock J. Physiol. Pathol. Gen, 9:620--635.
\end{thebibliography}

\end{document}

